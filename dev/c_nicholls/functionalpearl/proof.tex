\documentclass{journal}
\usepackage{ textcomp }
\usepackage{ifpdf}
\usepackage{ifthen}
\usepackage{mathtools}
\usepackage{subfigure}
\usepackage{epsfig}
\usepackage{ amssymb }

\title{\LARGE \bf
Waterfall proof
}
\author{Chris Nicholls
}
%\institute{Oxford University Computing Laboratory}

\begin{document}

Define a graph as $G = (N,E)$
where $N$ is the set of nodes and $E :: N\times N \times \mathbb{N}$ is the set of weighted edges
that must satisfy the following properties:
\begin{enumerate}
\item Transitivity. $(a,b,k) \in E  \iff (b,a,k) \in E$, if there is an edge from node a to node b, then there is an edge from node b to node a of the same weight.
\item Uniqueness.   $(a,b,k_1) \in E\ and\ (a,b,k_2) \implies k_1 = k_2$, there is at most one edge between two nodes.
\item We also require that no two edges are of equal weight, $(a_1,b_1,k) \in E\ and\ (a_2,b_2,k) \implies a_1 = a_2\ and\ b_1 = b_2$. This restriction ??? provide details.

\end{enumerate}

We define a regional-minima in the following way:
An edge $e = (a,b,k)$ is a regional minima iff
$\forall c \in E, l \in \mathbb{N} :
  (c,b,l) \in E \implies l \geq k\ and\
  (a,c,l) \in E \implies l \geq k $


\end{document}
