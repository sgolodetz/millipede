\documentclass{ouclprgsc}

%\addtolength{\textwidth}{40mm}
%\addtolength{\oddsidemargin}{-20mm}
%\addtolength{\textheight}{40mm}
%\addtolength{\topmargin}{-20mm}

\usepackage{ifpdf}
\usepackage{subfigure}
\usepackage{epsfig}

\title{\LARGE \bf
A simpler approach to waterfall
}
\author{Chris Nicholls, Irina Voiculescu and Stuart Golodetz
}
\institute{Oxford University Computing Laboratory}

\begin{document}

\maketitle
\pagestyle{empty}



Most applications of image segmentations of Computerised Tomography
(CT) scans require knowledge of where key anatomical features are
found in the image. 

@@@

 - segmentation is
 - waterfall algorithm for segmentation 
 - existing implementation (arguably overly complicated)
 - we have a better, simpler one
 - we will illustrate it on CT scans


Typically, the waterfall algorithm takes as input the output of
another algorithm, the watershed, introduced by Beucher and
Lantuejoul~\cite{beucher79}. 

 - say what the watershed does

@@@ A typical problem of the watershed algorithm is that it tends to
over-segment images significantly, leading to far more regions than
can be handled sensibly. This effect is due to problems such as
indistinct boundaries between features, the variation present between
different images, but mostly due to the algorithms not having any
knowledge of the context in which the segmentation takes place.

due to
 - noise in the image
 - spurious local minima


%---
%---
\begin{figure}
\centering
\ifpdf
        \subfigure{\epsfig{file=image0.png, width=.240\linewidth}}%
        \hspace{1mm}%
        \subfigure{\epsfig{file=image5.png, width=.240\linewidth}}%
\else
        % TODO
\fi
\caption{Example of oversegmentation, output by applying the watershed
  algorithm to an axial slice of a CT volume. The individual regions
  are small and do not correspond to any anatomic features.}
\label{fig:oversegmented}
\end{figure}

Figure~\ref{fig:oversegmented} illustrates (on a CT scan) the results
of a common segmentation algorithm, the watershed. The image is
clearly over-segmented and hence not much more useful than the
original for the purposes mentioned earlier. 

The waterfall algorithm is an iterative process which can extract
useful structure from this initial segmentation. The algorithm yields
a partition forest hierarchy, which is a comprehensive data structure
which can be used subsequently in the process of feature
identification.  Figure~\ref{fig:waterfall} illustrates the various
layers that result from applying the waterfall algorithm to the
segmentation shown in Figure~\ref{fig:oversegmented}.  Each iteration
of the algorithm yields a higher-level grouping of the regions in the
previous layer.

%---
\begin{figure}
\centering
\ifpdf
%        \subfigure{\epsfig{file=image0.png, height=.175\linewidth}}%
%        \hspace{4mm}%
%        \subfigure{\epsfig{file=image5.png, width=.240\linewidth}}%
%        \hspace{1mm}%
        \subfigure{\epsfig{file=image4.png, width=.260\linewidth}}%
        \hspace{1mm}%
        \subfigure{\epsfig{file=image2.png, width=.260\linewidth}}%
        \hspace{1mm}%
        \subfigure{\epsfig{file=image1.png, width=.260\linewidth}}%
\else
        % TODO
\fi
\caption{Hierarchy of segmentations produced by applying the waterfall
  algorithm to the output of the watershed illustrated in
  Figure~\ref{fig:oversegmented}, showing regions merging
  successively.}
\label{fig:waterfall}
\end{figure}
%---


Both the watershed and the waterfall algorithms are based on a
geographical metaphor. The image is regarded as a landscape, with each
grey value being proportional to the terrain height. 
%
The valleys are
in the darker areas, whereas the lighter areas are regarded as peaks.

The waterfall algorithm~\cite{beucher94,marcotegui} can then be imagined
as a flooding process. The water falls into (low) catchment basins and
gradually fills them up to the nearest boundaries, sometimes spilling
into adjacent regions. This process continues until the whole image
becomes a single basin. The intermediate stages of the process can be
regarded as intermediate segmentations of the image, with each basin
representing a region.

The traditional implementation of this algorithm \cite{marcotegui}
involves the construction of a Minimum Spanning Tree (MST) and the
gradual elision of some of its edges. 
Its nodes are initially the regions of the watershed output and its
edges are the lowest pass points on the boundaries between these
regions; the nodes and edges in subsequent layers are derived from
these intitial ones through a merging process.

The collection of regional minimum edges of a graph $G$ is a connected
subgraph of $G$ whose edges have the same weight as each other, and
whose adjacent edges in $G$ have strictly higher weights. The
waterfall algorithm relies heavily on finding these regional minimum
edges, eliding them and rebuilding the MST -- a process which not only
requires careful implementation of the MST but, more importantly, is
relatively complex and hard to implement.

In this paper we present a new data structure for the waterfall
algorithm that simplifies the process and improves efficiency compared
to current implementations. It is based on a recursive-tree data
structure and a recursive relation on the nodes rather than the
conventional iterative transformations.

The main advantage of our approach to the waterfall problem is that
the algorithm uses a single loop to walk the MST and is therefore
simpler to implement. For each iteration, it walks the MST bottom-up
in a single pass and merges regions that belong together. The
waterfall algorithm, thus improved, produces similar layers of
segmented images, combined in a hierarchical structure that can be
processed for feature identification.

A further advantage of our approach is that the algorithm can can be
written in pure functional style. In particular, we have implemented
it in Haskell. For this reason, the memory requirements are not
directly comparable to existing imperative implementations, but we are
about to integrate this new approach into an existing C++ code base.

We are also in the process of constructing a formal proof of
correctness, which we hope to present at a later date. We have tested
both algorithms on a number of small, measurable test cases and found
that they produce the same output. Empirical tests indicate that this
is also true of larger test cases, such as axial slices of CT volumes.

Production of partition forests in this manner also has many
applications outside of the field of medical imaging, for instance,
binary space partitioning in 3D map rendering for games.



\begin{thebibliography}{2}

\bibitem{beucher94}{Serge Beucher. {\em Watershed, hierarchical
    segmentation and waterfall algorithm}. In Mathematical Morphology
  and its Applications to Image Processing, Proc.\ ISMM 94, pages
  69-76, Fontainebleau, France, 1994. Kluwer Ac.\ Publ.}

\bibitem{beucher79}{S.\ Beucher, C.\ Lantuejoul. {\em Use of
    watersheds in contour detection}. International Workshop on image
  processing, real-time edge and motion detection/estimation, Rennes,
  France, Sept.\ 1979.}

\bibitem{marcotegui}{Beatriz Marcotegui and Serge Beucher, {\em Fast
    Implementation of Waterfall Based on Graph}. In Mathematical
  Morphology: 40 Years On, Springer Netherlands, 2005.}

\end{thebibliography}

\end{document}
